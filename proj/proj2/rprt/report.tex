%\documentclass[ 12pt, a4paper ]{article}
\documentclass[12pt, a4paper]{scrartcl}

%packages
\usepackage{physics}
\usepackage{amsmath}
\usepackage{algorithm}
\usepackage{algorithmic}
\usepackage{graphicx}
\usepackage{hyperref}
\usepackage{natbib}
\bibliographystyle{abbrvnat}

\title{Project 2: Classification and Regression}%, from linear and logistic regression to neural networks}
\subtitle{from linear and logistic regression to neural networks}

\author{Anders Eriksen}
\date{}

%//////////////////////
\begin{document}
%=====
%table of contents, list of figs and content
\maketitle
\tableofcontents
%=====
\begin{abstract}
    The main summary of the work
\end{abstract}

\section{introduction}
    aims and rationale of the physics case, what you've done as well as a brief summary of 
the report \\
    o   Motivate the reader\\
    o   What done\\
    o   Structure of report\\
\section{Methods}
    Theoretical models and technicalities. \\
    o   describe methods and algorithms\\
    o   explain implementation of methods and say something about the structure of
        algorithm and present parts of code\\
    o   Plug in some calculations to test code, such as selected runs used to validate and
        verify results. Latter extremely important. Reader needs to know that your code
        reproduces selected benchmarks and reproduces previous results, either numerical 
        and/or well-known closed-form expressions. \\

   The main aim here is to study both classification and regression problems. With an intention to 
reuse the regression algorithms studied in project 1 of the course, \cite{project1}. Further we include logistic regression
for classification problems. Additinally implementing multilayer perception code for both regression and 
classification problems. Mean squared error, cross validation algorithm as well as the R2 or the accuracy 
score for classification can be utilized here. \\

    The data sets used are terrain data from the previous project as well as UCI's credit card data set. These 
are for the regression and classification respectively. %THESE DATA SETS ARE MERELY SUGGESTIONS. PERHAPS SOMETHING
%ON THE ISING MODEL, OR QUANTUM DOTS. An example for regression is the so-called "Wisconsin Cancer data" - a binary
%problem, beningn or malignant tumor - This set has several examples.
\\

    The first point of order is to familiarize ourselves with the data. Import and organize the set to gain some 
idea about the shape, size and layout of the data. Following this comes defining the cost function and design matrix.
These should be mostly the same as during the first project, \cite{project1}. A gradient descent method would also likely be 
helpfull here, as well as further on in the project. This method could also be compared to scikitlearn's modules. \\

    In order to measure the preformance in the classification problem a so-called \textit{Accuracy score} is used. This is 
the number of correct guessed targets $t_i$ over the total number of targets. A perfect classifier would yield an accuracy of 1.
\begin{align}
    Accuracy \,=\, \frac{ \sum_{i_1}^n I\qty( t_i = y ) }{ n }.
\label{eq:accuracy}
\end{align}
I here is an indicator function returning 1 if $t_i = y$ and 0 otherwise in this instance of a binary classification. $y_i$ 
is the output of the Logistic Regression code. The accuracy can be compared with the scikitlearn for performance benchmarking. \\


    Once the groundwork is laid, we can begin work on a feed forward neural network. The back propagation algorithm based on course
slides, \cite{slides}, 
\begin{algorithm}
\caption{back propagation algorithm}
\begin{algorithmic}
\FOR{ $i=0$ \TO $i=n$ }
\STATE Iterate
\ENDFOR
\end{algorithmic}
\label{algo:backpropagation}
\end{algorithm}

    Once the algorithm is in place, a discussion of cost function is advised. 


\section{Results}
    The results and discussion of such\\
    o   Present results \\
    o   Give critical discussion of you work \& place it in correct context\\
    o   Relate work to other calculations/studies\\
    o   Reader should be able to reproduce calculations should they wish to do so. 
        All input variables should be properly explained. \\
    o   Make sure figures and tables contain enough information in their captions. 
        Axis labels, etc. A reader should be able to get a first impression of the work
        by purely studying the figures and tables. \\
\section{Conclusion}
    Conclusions and perspectives\\
    o   State main findings and interpretations \\
    o   Try as far as possible to present perspectives for future work. \\
    o   Try to discuss the pros and cons of the methods and possible improvements. \\
\section{Appendix}
    any extra material \\
    o   Additional calculations used to validate code. \\
    o   Selected calculations. Can be listed with few comments. \\
    o   Listing of code if necessary. \\
    Consider moving parts from methods to appendix. A webpage is also an appropriate place
    for a lot of this type of info. \\\\

%bibliography
 %References
    o   Always reference material you base your work on, either scientific articles/reports
        or books. \\
    o   Refer to articles as: name(s) of author(s), journal, volume(Bold), page and year in
        parenthesis. \\
    o   Refer to bookds as: name(s) of author(s), title of book, publisher, place and year, 
        eventual page numbers. \\

\bibliography{proj2bib}
\end{document}
%/////////////////////


