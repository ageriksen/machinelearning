\documentclass[
    12pt, %main doc font size
    a4paper, 
    ]{article}

%\title{\normalfont\spacedallcaps{Title here}}
%\subtitle{subtitle here}

%\author{Barholomew Finch}
%\date{}

%//////////////////////
\begin{document}
%=====
%table of contents, list of figs and content
%\maketitle
%\tableofcontents
%\listoffigures
%\listoftables
%=====
\section{Design Matrix}

\subsection{Initial description}
The design matrix is what relates the criteria $\vec{\beta}$ to a prediction of outcomes, $\vec{\tilde{y}}$.
\begin{equation}
    \vec{\tilde{y}} = \hat{X}\,\vec{\beta}
\end{equation}
And through manipulations of this relation, one can tweak $\beta$ to arive at a good prediction for $y$ on a 
statistical level. The accuracy of the prediction can then be measured through analysing the average errors between
prediction and the known \it{training values}. 

%notes:
% The design matrix is composed of a set of independent variables of various "objects".
% Each row pertains to 1 object. Each collumn corresponds to a different independent variable, with their
%   respective value for the object on the row. 
\end{document}
%/////////////////////
